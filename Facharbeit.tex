\documentclass[a4paper,12pt]{article}
\usepackage[T1]{fontenc}
\usepackage[utf8]{inputenc}
\usepackage{lmodern}
\usepackage[german]{babel}
\usepackage{geometry}
\usepackage[onehalfspacing]{setspace}%für 1,5-fachen Zeilenabstand
\usepackage{hyperref}
%TODO set font
\pagestyle{headings}

\title{Der Nutzen des Konzeptes Open Source bei der Softwareentwicklung am Beispiel des Linux-Kernels}
\date{}
\begin{document}
\maketitle
\thispagestyle{empty}
\begin{center}
  \LARGE Autor:    Josua Brandhofer
\end{center}
\begin{verbatim}
\end{verbatim}
\begin{center}
  \large\textbf{In:   Informatik (GK1)}
\end{center}
\begin{verbatim}
\end{verbatim}
\begin{center}
  \large\textbf{Betreuungs-/Kurslehrer:    Herr Berger}
\end{center}
\begin{verbatim}
\end{verbatim}
\begin{center}
  \large\textbf{Name der Schule:    Siegtal-Gymnasium Eitorf}
\end{center}
\begin{verbatim}
\end{verbatim}
\begin{center}
  \large\textbf{Jahreszahl der Entstehung:    2015}
\end{center}

\newpage
\newgeometry{left=4cm,right=2cm,top=2.5cm,bottom=2.5cm}
\tableofcontents
\newpage
\section{Einleitung}
Was ist Open Source?, Wieso bedeutsam, sich damit auseinanderzusetzen, Methoden laberlljdfawefejklcmjcijeknjnfwelavnejkfenwfjwenfwenefnhfujnjh
\section{Definition OpenSource und Abgrenzung}
\section{Was ist der Linux-Kernel?}
\section{Geschichte von Open Source und die des Linux-Kernels}
\section{Nutzen für die Beteiligten}
\subsection{Nutzen für den Konsumenten}
\subsubsection{Ein Fallbeispiel}
\subsection{Nutzen für die Programmierer}
\subsection{Nutzen für Firmen}
\section{Kritik an Open Source}
\section{Schlussteil}
\newpage
\newgeometry{left=2cm,right=2cm,top=2.5cm,bottom=2.5cm}
\section{Literaturverzeichnis}
\begin{verbatim}
\end{verbatim}
Prof. Dr. Schwalbe, Ulrich: \textbf{Open Source Software - Eine wirtschaftstheoretische}
\newline
\textbf{Analyse}, überarbeitete Version von Fahrig, Thomas -
\newline
\url{http://www.mafabo.de/thomas/da/oss-aktuell.php} (08.08.2008)
\begin{verbatim}
\end{verbatim}
Wikipedia: \textbf{Open Source} - \url{http://de.wikipedia.org/wiki/Open_Source} (07.01.2015)
\begin{verbatim}
\end{verbatim}
Wikipedia: \textbf{Freie Software} - \url{http://de.wikipedia.org/wiki/Freie_Software} (31.01.2015)
\begin{verbatim}
\end{verbatim}
Wikipedia: \textbf{Linux} - \url{http://de.wikipedia.org/wiki/Linux} (14.01.2015)
\begin{verbatim}
\end{verbatim}
Wikipedia: \textbf{Geschichte von Linux} - \url{http://de.wikipedia.org/wiki/Geschichte_von_Linux} (28.01.2015)
\begin{verbatim}
\end{verbatim}
Wikipedia: \textbf{Linux (Kernel)} - \url{http://de.wikipedia.org/wiki/Linux_(Kernel)} (07.01.2015)
\begin{verbatim}
\end{verbatim}
Stallman, Richard: \textbf{Warum Open Source das Ziel von Freie Software verfehlt}, deutsche Übersetzung von Gehring, Robert; Kohne,Joerg - 
\newline
\url{https://www.gnu.org/philosophy/open-source-misses-the-point.de.html} (30.10.2014)
\begin{verbatim}
\end{verbatim}
Wilkens, Andreas: \textbf{Studie übt harte Kritik an Open-Source-Software} In: heise online - 
\newline
\url{http://heise.de/-97045} (15.04.2004)
\end{document}
