\documentclass[a4paper,12pt]{article}
\usepackage[T1]{fontenc}
\usepackage[utf8]{inputenc}
\usepackage{lmodern}
\usepackage[german]{babel}
\usepackage{geometry}
\usepackage[onehalfspacing]{setspace}%für 1,5-fachen Zeilenabstand
%TODO set font
\pagestyle{headings}

\title{Der Nutzen des Konzeptes Open Source bei der Softwareentwicklung am Beispiel des Linux-Kernels}
\date{}
\begin{document}
\maketitle
\thispagestyle{empty}
\begin{center}
  \LARGE Autor:    Josua Brandhofer
\end{center}
\begin{verbatim}
\end{verbatim}
\begin{center}
  \large\textbf{Im Fach:    Informatik}
\end{center}
\begin{verbatim}
\end{verbatim}
\begin{center}
  \large\textbf{Betreuungslehrer:    Herr Berger}
\end{center}
\begin{verbatim}
\end{verbatim}
\begin{center}
  \large\textbf{Name der Schule:    Siegtal-Gymnasium Eitorf}
\end{center}
\begin{verbatim}
\end{verbatim}
\begin{center}
  \large\textbf{Abgabedatum:    \today}
\end{center}

\newpage
\newgeometry{left=4cm,right=2cm,top=2.5cm,bottom=2.5cm}
\tableofcontents
\newpage
\section{Einleitung}
Was ist Open Source?, Wieso bedeutsam, sich damit auseinanderzusetzen, Methoden laberlljdfawefejklcmjcijeknjnfwelavnejkfenwfjwenfwenefnhfujnjh
\section{Definition OpenSource und Abgrenzung zu freier und proprietärer Software}
\section{Was ist der Linux-Kernel?}
\section{Geschichte von Open Source}
\subsection{Die Anfänge}
\subsection{Der Beginn von proprietärer Software}
\subsection{OpenSourceInitiative}
\section{Nutzen für die Beteiligten}
\subsection{Nutzen für den Konsumenten}
\subsubsection{Ein Fallbeispiel}
\subsection{Nutzen für die Programmierer}
\subsection{Nutzen für Firmen}
\section{Schlussteil}
\section{Literaturverzeichnis}
\end{document}
