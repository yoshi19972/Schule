\documentclass[a4paper,12pt]{article}
\usepackage[T1]{fontenc}
\usepackage[utf8]{inputenc}
\usepackage{lmodern}
\usepackage[german]{babel}
\usepackage{geometry}
\usepackage[onehalfspacing]{setspace}%für 1,5-fachen Zeilenabstand
\usepackage{hyperref}
%TODO set font
\pagestyle{headings}

\title{Der Nutzen des Konzeptes Open Source bei der Softwareentwicklung am Beispiel des Linux-Kernels}
\date{}
\begin{document}
\maketitle
\thispagestyle{empty}
\begin{center}
  \LARGE Autor:    Josua Brandhofer
\end{center}
\begin{verbatim}
\end{verbatim}
\begin{center}
  \large\textbf{In:   Informatik (GK1)}
\end{center}
\begin{verbatim}
\end{verbatim}
\begin{center}
  \large\textbf{Betreuungs-/Kurslehrer:    Herr Berger}
\end{center}
\begin{verbatim}
\end{verbatim}
\begin{center}
  \large\textbf{Name der Schule:    Siegtal-Gymnasium Eitorf}
\end{center}
\begin{verbatim}
\end{verbatim}
\begin{center}
  \large\textbf{Jahreszahl der Entstehung:    2015}
\end{center}

\newpage
\newgeometry{left=4cm,right=2cm,top=2.5cm,bottom=2.5cm}
\tableofcontents
\newpage
\section{Einleitung}
Computer sind aus dem heutigen Leben kaum noch wegzudenken. Sie werden zur Freizeitgestaltung oder bei der Arbeit genutzt. In den Industrieländer hat beinahe jeder Haushalt einen Computer. Hinter dem Computer steckt nicht nur die bloße Hardware. Noch viel wichtiger dabei ist die Software. Hinter der Softwareentwicklung oder den IT-Dienstleistungen steckt ein großer Markt, welcher immer wichtiger wird. Solche Software kann ein Spiel, ein Programm zur Steuerung komplexer Anlagen oder ein ganzes Betriebssystem sein. Als bekanntestes Betriebssystem gilt dabei das proprietäre Betriebssystem „Windows“ von Microsoft. Dieses wird kommerziell vertrieben und wird auf den meisten Privat Computern vorinstalliert. Daneben gibt es aber auch die sogenannte Open Source Software. Prominente Beispiele dafür sind z.~B. der Internet Browser~„Mozilla Firefox“ oder das Betriebssystem~„Linux” mit all seinen verschiedenen Distributionen. Diese können von jedem, anders als bei „Windows“, unentgeltlich genutzt und weiterentwickelt werden. Meistens wird „Linux“ aber eher versteckt genutzt. So nutzt man eigentlich täglich Linux wenn man „Google.com“ benutzt, da die Server von Google mit einem Linux-Betriebssystem laufen. Auch bei Supercomputern wird Linux sehr häufig genutzt. Aktuell werden 94,6\% der Supercomputer zumindest teilweise mit Linux betrieben\footnote{vgl.: Wikipedia; Supercomputer; 2015; Abschnitt: Betriebssysteme}. Bei diesem häufigen Vorkommen kam auch Herr Prof. Dr. Schwalbe zu dem Schluss: „Somit wird deutlich, dass auch Open Source Software für viele Bereiche des Lebens bedeutend ist und aus der heutigen, durch Informationstechnologien geprägten Welt, kaum noch heraus zu denken ist.“ (Schwalbe, 1. Einleitung). Bei so einem großen Erfolg von Open Source Software stellt sich die Frage: Inwiefern hilft das Konzept von Open Source Software ein so gutes Produkt\footnote{So gut, dass Microsoft es als große Bedrohung ansieht(siehe auch Halloween-Dokumente)} zu erstellen, welches häufig kostenlos ist? Inwiefern ist dieses Konzept vom Ergebnis her besser als proprietäre Software?\\
In dieser Facharbeit möchte ich mich mit dieser Frage auseinandersetzen und dabei mehrere Texte untersuchen. Dabei werde ich mich am Beispiel des Linux-Kernels orientieren. Zu Anfang werde ich erstmal eine kurze Definition von Open Source Software geben und sie dabei von anderen prominenten Softwaretypen abgrenzen. Zusätzlich gebe ich eine kurze Darstellung des Linux-Kernels. Darauf werde ich im 4. Kapitel die Geschichte von Open Source Software erörtern, wobei ich mich auf die des Linux-Kernels konzentriere. Im nachfolgenden Kapitel werde ich den Nutzen des Open Source Konzepts für verschiedene Beteiligte untersuchen. Dies wären das Projekt an sich, die Konsumenten, für die die Software gemacht wurde, die teilnehmenden Programmierer und schlussendlich die Firmen, die die Open Source Software nutzen oder mitentwickeln. Im letzten Kapitel werde ich mich mit Kritik an Open Source Software auseinandersetzen.
\section{Definition Open Source Software und Abgrenzung}
Open Source Software ist, wenn man es wörtlich übersetzt, Software, deren Quelltext offengelegt ist. Man sollte dies aber nicht mit gemeinfreier Software, welche nicht lizenziert wird und allen kostenlos zugänglich ist, verwechseln. Denn Open Source Software wird sehr wohl lizenziert. Dabei gelten laut der Open Source Initiative (im folgenden OST) für Lizenzierung im Sinne von Open Source folgende 3 Grundmerkmale (laut Wikipedia, Open Source, Abschnitt: Definition der OSI): 1. „Die Software (d. h. der Quelltext) liegt in einer für den Menschen lesbaren und verständlichen Form vor.“, 2. „Die Software darf beliebig kopiert, verbreitet und genutzt werden.“ und 3. „Die Software darf verändert und in der veränderten Form weitergegeben werden.“\footnote{Bei mancher Software wird ergänzend zum 3. Merkmal noch weiterführend das Copyleft-Prinzip angewendet, um auch weiterhin die Rechte des Nutzers zu wahren.}. Also gibt Open Source Software seinen Nutzern weitaus mehr Rechte als proprietäre Software, welche ihren Quelltext meistens geheimhällt. Diese weitergehenden Rechte bedeuten aber nicht, dass die Software unbedingt kostenlos angeboten wird. Die mögliche Missinterpretation des Wortes frei in der zur Open Source Software weitgehend identischen freien Software hat primär zur Entstehung der OSI beigetragen. Daher stammt auch die Aussage der Free Software Foundation (FSF) über die Bedeutung des Wortes „frei“: „free speech, not free beer“(freie Meinungsäußerung, nicht Freibier“). Nichtsdestotrotz wird trotzdem sehr viele Open Source Software kostenlos angeboten\footnote{Das Unternehmen ID Software geht dabei einen interessanten Weg. Sie machen den Quelltext ihrer Spiele öffentlich zugänglich, z.~B. auf GitHub, zugänglich und verlangen nur Geld für die graphischen und musikalischen Inhalte.}. Die meisten Open Source Projekte werden unter der von der FSF erstellten GNU General Public License veröffentlicht.\\
Open Source muss man zwangsläufig zu der sehr ähnlichen freien Software abgrenzen. Diese sind zwar sehr ähnlich, nur garantiert freie Software mehr Freiheiten für den Nutzer als Open Source Software\footnote{Was nicht bedeutet, dass Open Source Software diese nicht bieten kann.}. Der größte Unterschied besteht aber in der Bedeutung der Begriffe und in den Werten, den sie vermitteln: Open Source zeigt die freie Verfügbarkeit des Quelltextes. Auch stellt die OSI primär den praktischen Nutzen dieser Entwicklungsmethode heraus und zeigt eher technische und wirtschaftliche Vorteile. Die freie Software nach der FSF vertritt eher die Rechte der Nutzer an der Software und zeigt auch sozial ethische Aspekte auf. So geht es ihr bei der Ablehnung proprietärer Software nicht bloß darum, wie gut sie trotz nicht öffentlichen Quelltextes auch sein mag, sondern behauptet, „dass proprietäre Software allein schon aus moralischen Gründen abzulehnen sei\footnote{Wikipedia, Freie Software, Abschnitt: Open Source}“. Genauer auf freie Software und alle Unterschiede zu Open Source Software einzugehen würde den Rahmen dieser Facharbeit sprengen.
\section{Was ist der Linux-Kernel?}%Quellen: wikipedia: Linux; Linux(Kernel)
„Linux ist ein Betriebssystem-Kernel. [...] Der Kernel wird in einer Vielzahl von Betriebssystemen genutzt, die oft selbst als Linux bezeichnet werden.“ (Wikipedia, Linux(Kernel)). Einer der Hauptentwickler des Linux-Kernels ist Linus Torvalds. Als Kernel ist der Linux-Kernel eines der wichtigsten Teile des Betriebssystem. Er sorgt dafür, dass eine einheitliche Schnittstelle für die Software zur Verfügung steht, welche Hardware unabhängig ist. Der in weitgehend in C geschriebene Kernel ist ein strikt monolithischer Kernel\footnote{D. h. werden der Quellcode und alle Treiber in das Kernelimage kompiliert. Das Kernelimage ist der ausführbare Kernel.}. Des weiteren bietet der Kernel 4 verschiedene Schnittstellen: die in jedem Fall stabile externe Programmierschnittstelle, die nicht stabilitätsgarantierte interne Programmierschnittstelle, die „Linux Standard Base[, welche eine Binärschnittstelle ist,] soll es ermöglichen kommerzielle Programme unverändert zwischen Linux Betriebssystemen zu portieren“ (Wikipedia, Linux(Kernel)) und schlussendlich die nicht stabile interne Binärschnittstelle. Obwohl es eigentlich nicht so geplant war, hat sich der Linux-Kernel weitestgehend in Richtung eines portierbaren Kernels entwickelt und gehört zu einem der am häufigsten portierten Systemen. Der Kernel ist seit 1992 unter der freien GPL lizenziert.
Als Linux bezeichnet man Betriebssysteme, die auf dem Linux-Kernel und im Wesentlichen auf GNU-Software basieren. Auf Computern kommt Linux in Form einer Distribution vor. „Eine Distribution fasst den Linux-Kernel mit verschiedener Software zu einem Betriebssystem zusammen, das für die Endnutzung geeignet ist.“ (Wikipedia, Linux). Die hohe Anzahl an Linux-Distributionen bietet dem Endnutzer die Möglichkeit, sein auf ihn persönlich zugeschnittenes Linux zu erhalten. Das auffälligste Merkmal an Linux ist seine hohe Sicherheit vor Viren und dergleichen. Dies wird durch eine strenge Unterteilung der Zugriffsrechte erreicht, so dass die Verunreinigung des gesamten Systems schwer ist. Auch die schnelle Aktualisierung hilft, Sicherheitslücken zu schließen.
\newpage
\section{Geschichte von Open Source und die des Linux-Kernels}%Quellen: wikipedia: Open Source#Geschichte; Geschichte von Linux#Entwicklungen im Vorfeld; Freie Software#Geschichte; Schwalbe:2.3; Linux: wikipedia: Linux; Linux(Kernel)#Entwicklungsprozess; Geschichte von Linux
%\section{Organisation von Open Source Projekten}%Sollte ich noch Platz haben Quellen: Schwalbe: 3
\section{Nutzen für die Beteiligten}%Quellen: wikipedia: Open Source#Motivation; Schwalbe:2.5; 4; (5); 7; Kernelversionen
\subsection{Nutzen für das Projekt}
\subsection{Nutzen für den Konsumenten}%Quellen: Freie Software#Freie Software aus gesellschaftlicher Sicht
\subsubsection{Ein Fallbeispiel}%Quellen: Eigene Erfahrung: Problem mit resume from suspend to RAM bei Linux 3.17.4
\subsection{Nutzen für die Programmierer}%Quellen: 
\subsection{Nutzen für Firmen}%Quellen: Schwalbe: 6; ZDNet
\section{Kritik an Open Source}%Quellen: heisse online; gnu project; wikipedia: Freie Software#Gefahren für freie Software; Open Source#Probleme
Durch das GNU-Projekt; durch die Studie; durch Microsoft(wiedersprechend!)
\section{Schlussteil} %Schwalbe: 8
\newpage
\newgeometry{left=2cm,right=2cm,top=2.5cm,bottom=2.5cm}
\section{Literaturverzeichnis}
\begin{verbatim}
\end{verbatim}
Prof. Dr. Schwalbe, Ulrich: \textbf{Open Source Software - Eine wirtschaftstheoretische}
\newline
\textbf{Analyse}, überarbeitete Version von Fahrig, Thomas -
\newline
\url{http://www.mafabo.de/thomas/da/oss-aktuell.php} (08.08.2008)
\begin{verbatim}
\end{verbatim}
Wikipedia: \textbf{Open Source} - \url{http://de.wikipedia.org/wiki/Open_Source} (07.01.2015)
\begin{verbatim}
\end{verbatim}
Wikipedia: \textbf{Freie Software} - \url{http://de.wikipedia.org/wiki/Freie_Software} (31.01.2015)
\begin{verbatim}
\end{verbatim}
Wikipedia: \textbf{Linux} - \url{http://de.wikipedia.org/wiki/Linux} (14.01.2015)
\begin{verbatim}
\end{verbatim}
Wikipedia: \textbf{Geschichte von Linux} - \url{http://de.wikipedia.org/wiki/Geschichte_von_Linux} (28.01.2015)
\begin{verbatim}
\end{verbatim}
Wikipedia: \textbf{Linux (Kernel)} - \url{http://de.wikipedia.org/wiki/Linux_(Kernel)} (07.01.2015)
\begin{verbatim}
\end{verbatim}
Stallman, Richard: \textbf{Warum Open Source das Ziel von Freie Software verfehlt}, deutsche Übersetzung von Gehring, Robert; Kohne,Joerg - 
\newline
\url{https://www.gnu.org/philosophy/open-source-misses-the-point.de.html} (30.10.2014)
\begin{verbatim}
\end{verbatim}
Wilkens, Andreas: \textbf{Studie übt harte Kritik an Open-Source-Software} In: heise online - 
\newline
\url{http://heise.de/-97045} (15.04.2004)
\begin{verbatim}
\end{verbatim}
Vaughan-Nichols, Steven J.:\textbf{Why Microsoft loves Linux} In: ZDNet -
\newline
\url{http://www.zdnet.com/article/why-microsoft-loves-linux/} (29.10.2014)
\begin{verbatim}
\end{verbatim}
Wikipedia: \textbf{Supercomputer} - \url{http://de.wikipedia.org/wiki/Supercomputer#Betriebssysteme} (13.02.2015)
\end{document}
